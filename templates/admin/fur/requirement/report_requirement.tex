\documentclass[10pt]{article}

\usepackage[utf8]{inputenc}
\usepackage{graphicx,url}
\usepackage[brazil]{babel}
\usepackage[linkcolor=blue,citecolor=blue,urlcolor=blue,colorlinks,pdfpagelabels]{hyperref}
\usepackage[a4paper,top=3.5cm,left=3cm,right=3cm,bottom=2.5cm]{geometry}
\usepackage{longtable}

\title{Relatório de Requisitos -- draft} %mude para o titulo do seu relatorio
\author{RiSE/Medicware} %mude para os autores do relatorio

\begin{document} 

\maketitle

\newpage

\tableofcontents

\newpage


	\setcounter{section}{ {{requirement.id|add:'-1'}}}
	\section{~REQ:{{requirement.name}}}
	\label{req{{requirement.id}}}
	
	\begin{longtable}{|p{2.5cm}|p{12cm}|}
	\hline
	\textbf {Descrição:} & {{requirement.description}} \\
	\hline
    \textbf{Variabilidade:} & {{requirement.variabilityType}}\\
	\hline
	\textbf{Prioridade:} & {{requirement.priority}}\\
	\hline
	\textbf{Features:} & 
	
	  {{feat.name}} [{{feat.id}}]
	\\
	
	  \hline
	  \textbf{Pontos de Variação:} &
	 
	  
	    \textbf{ {{varpoint.name}}: }{{varpoint.description}}\newline
	    \emph{Cardinalidade mínima:} {{varpoint.cardinalityMin}}\newline
	    \emph{Cardinalidade máxima:} {{varpoint.cardinalityMax}}
		
		
		  \textbf{ Variantes:}
	      
	        \textbf{ {{variant.name}}}: {{variant.description}}.
	        
	        \newline
	        \emph{Feature:} {{variant.feature.name}}
	        
	      
	    
	  \\
	
	\hline
	\end{longtable}



\end{document}
